\documentclass[a4paper]{article}
\usepackage[francais]{babel}
\usepackage[T1]{fontenc}  
\usepackage[utf8x]{inputenc} 
\usepackage{amsmath}
\usepackage{amsfonts}
\usepackage{amssymb}
\usepackage{graphicx}
\usepackage[a4paper, left=2cm, right=2cm, top=2.5cm, bottom=2cm, headheight=40pt]{geometry}
\usepackage{color}
\usepackage{mathtools}
\usepackage{algcompatible}
\usepackage{algorithm}
\usepackage{algorithmicx}
\usepackage{algpseudocode}
\usepackage[dvipsnames]{xcolor}
\usepackage{tikz}
\usetikzlibrary{shapes}
\usepackage[square,sort,comma,numbers]{natbib}
%\usepackage{emptypage}
\usepackage{hyperref}
\hypersetup{
    colorlinks,
    citecolor=black,
    filecolor=black,
    linkcolor=black,
    urlcolor=black
}
\newcount\colveccount
\newcommand*\colvec[1]{
        \global\colveccount#1
        \begin{pmatrix}
        \colvecnext
}
\def\colvecnext#1{
        #1
        \global\advance\colveccount-1
        \ifnum\colveccount>0
                \\
                \expandafter\colvecnext
        \else
                \end{pmatrix}
        \fi
}
\title{Hashing Functions}
\author{}


\begin{document}
\begin{titlepage}
  \begin{sffamily}
  \begin{center}

    Univesité de Bordeaux\\ Sciences \& Technologie\\
           351 Cours de la Liberation\\33400 Talence\\[1em]
            \textbf{\underline{- Projet Cryptanalyse -}}\\[1.5cm]
           
    \includegraphics[scale=0.11]{UB.jpg}
    \\[3cm]

    { \huge \bfseries Projet sur la cryptanalyse de A5/2\\[0.5cm] }
    \begin{flushright}
      \bfseries {Amelie GUEMON \& Ida TUCKER\\[1em]Master 2 CSI --- Cryptologie \& Sécurité Informatique}\\[6.6cm]
    \end{flushright}
    
   
    \today

  \end{center}
  \end{sffamily}
\end{titlepage}

%% QUESTION 1
\section{Implémentation d du chiffrement A5/2 }
Implémentation du chiffrement A5/2, cf. a5-2.sage

%% QUESTION 2
\section{Question 2}


Avant le $1^{er}$ passage par la fonction A5/2-step on a: 
\begin{itemize}
\item $R_1 = (x_0, \ldots, x_{18})$
\item $R_2 = (x_{19}, \ldots, x_{40})$
\item $R_1 = (x_{41}, \ldots, x_{63})$
\end{itemize}

Notons, au $k^{ième}$ passage par la fonction A5/2-step:
\begin{itemize}
\item $R_1 = (x^{\lbrack k\rbrack}_0, \ldots, x^{\lbrack k \rbrack}_{18})$
\item $R_2 = (x^{\lbrack k \rbrack}_{19}, \ldots, x^{\lbrack k \rbrack}_{40})$
\item $R_3 = (x^{\lbrack k \rbrack}_{41}, \ldots, x^{\lbrack k \rbrack}_{63})$
\end{itemize}

Montrons qu'on peut exprimer le contenu des registres $R_1$, $R_2$, $R_3$ au moyen d'équations linéaires en les $x_i$ par récurrence.

Il est clair qu'à l'étape zero (avant le premier passage par A5/2-step), la propriété est vraie.
Supposons la vraie au rang $k$ pour $k \geq 0$.
On a donc $x^{\lbrack k\rbrack}_0, \ldots, x^{\lbrack k \rbrack}_{63}$ qui s'expriment comme équation linéaire des $x_i$.
On passe alors dans la fonction A5/2-step:
\begin{itemize}
\item Si m = $R_{4,6}$ on met à jour le $LFSR_1$, et $R_1$ devient:
$$R_1 = (x^{\lbrack k+1\rbrack}_0, \ldots, x^{\lbrack k+1 \rbrack}_{18}) = (x^{\lbrack k\rbrack}_1, x^{\lbrack k\rbrack}_2, \ldots, x^{\lbrack k \rbrack}_{18}, x^{\lbrack k\rbrack}_0 + x^{\lbrack k\rbrack}_1 + x^{\lbrack k\rbrack}_2 + x^{\lbrack k\rbrack}_5)$$
\item Si m = $R_{4,13}$ on met à jour le $LFSR_2$, et $R_2$ devient:
$$R_2 = (x^{\lbrack k+1\rbrack}_{19}, \ldots, x^{\lbrack k+1 \rbrack}_{40}) = (x^{\lbrack k\rbrack}_{20},x^{\lbrack k\rbrack}_{21}, \ldots, x^{\lbrack k+1 \rbrack}_{40}, x^{\lbrack k\rbrack}_{19} + x^{\lbrack k\rbrack}_{20})$$
\item Si m = $R_{4,9}$ on met à jour le $LFSR_3$, et $R_3$ devient:
$$R_3 = (x^{\lbrack k+1\rbrack}_{41}, \ldots, x^{\lbrack k+1 \rbrack}_{63}) = (x^{\lbrack k\rbrack}_{42}, x^{\lbrack k\rbrack}_{43}, \ldots, x^{\lbrack k\rbrack}_{63},x^{\lbrack k\rbrack}_{41} + x^{\lbrack k\rbrack}_{42}+ x^{\lbrack k\rbrack}_{43}+ x^{\lbrack k\rbrack}_{56})$$
\end{itemize}

On a donc le dernier élément de chaque registre au $(k+1)^{ième}$ passage qui est une somme des contenus des registre au $k^{ième}$ passage.
C'est bien une équation linéaire en les $x_i$ par linéarité de la somme.
Les autres éléments des registres sont les éléments de l'étape précédente décalés. C'est à dire $x^{\lbrack k+1\rbrack}_{i} = x^{\lbrack k\rbrack}_{i+1}$ pour $ i \neq 18, 40, 63$.

On a bien montré que durant toutes les étapes de la production de suite chiffrante de A5/2, on peut exprimer les contenus des registres $R_1$, $R_2$, $R_3$ au moyen d'équations linéaires en les $x_i$. 

%% QUESTION 1
\section{Expression des registres au moyen d'équations linéaires}
Expression des Registres $R_1$, $R_2$, $R_3$ au moyen d'équations linéaires en les $x_i$ en sage, cf. a5-2.sage

%% QUESTION 4
\section{Expression des bits de z comme équations quadratiques en les $x_i$: théorique}
\begin{table}[h!]
\centering
\begin{tabular}{l l l|c}
a & b & c & Maj(a,b,c) \\\hline
0 & 0 & 0 & 0  \\
0 & 0 & 1 & 0 \\
0 & 1 & 0 & 0 \\
0 & 1 & 1 & 1 \\
1 & 0 & 0 & 0  \\
1 & 0 & 1 & 1 \\
1 & 1 & 0 & 1 \\
1 & 1 & 1 & 1 \\

\end{tabular}
\caption{\label{tab:widgets}Table de vérité de Maj}
\end{table}

On voit, d'après sa table de vérité, que la forme algébrique normale de la fonction booléenne Maj est:
$$f(a,b,c) = ab \oplus ac \oplus bc$$
\paragraph{}
Au $(99+i)^{ième}$ passage par la fonction A5/2-step dans la production, on produit le $(i)^{ième}$ bit de la suite chiffrante: 

\begin{equation}
\begin{aligned}
  &z_i =& x^{\lbrack i + 99\rbrack}_0 + \mbox{Maj}(x^{\lbrack i+99\rbrack}_{3}, x^{\lbrack i+99\rbrack}_{4} + 1, x^{\lbrack i+99\rbrack}_{6})) \\
  & &+ x^{\lbrack i + 99\rbrack}_{19} + \mbox{Maj}(x^{\lbrack i+99\rbrack}_{27}, x^{\lbrack i+99\rbrack}_{24} + 1, x^{\lbrack i+99\rbrack}_{31})) \\
  & &+ x^{\lbrack i + 99\rbrack}_{41} + \mbox{Maj}(x^{\lbrack i+99\rbrack}_{45}, x^{\lbrack i+99\rbrack}_{50} + 1, x^{\lbrack i+99\rbrack}_{47})) \\
\end{aligned}
\end{equation}

On peut remplaçer la fonction Maj par sa forme normale algébrique et on obtient:

\begin{equation}
\begin{aligned}
  &z_i =& x^{\lbrack i + 99\rbrack}_0 + (x^{\lbrack i+99\rbrack}_{3}( x^{\lbrack i+99\rbrack}_{4} + 1) + x^{\lbrack i+99\rbrack}_{3} x^{\lbrack i+99\rbrack}_{6} + ( x^{\lbrack i+99\rbrack}_{4} + 1) x^{\lbrack i+99\rbrack}_{6}) \\  
  & &+ x^{\lbrack i + 99\rbrack}_{19} + (x^{\lbrack i+99\rbrack}_{27}(x^{\lbrack i+99\rbrack}_{24} + 1) + x^{\lbrack i+99\rbrack}_{27} x^{\lbrack i+99\rbrack}_{31} + (x^{\lbrack i+99\rbrack}_{24} + 1)x^{\lbrack i+99\rbrack}_{31}) \\  
  & &+ x^{\lbrack i + 99\rbrack}_{41} + (x^{\lbrack i+99\rbrack}_{45} (x^{\lbrack i+99\rbrack}_{50} + 1) + x^{\lbrack i+99\rbrack}_{45} x^{\lbrack i+99\rbrack}_{47} + (x^{\lbrack i+99\rbrack}_{50} + 1)x^{\lbrack i+99\rbrack}_{47}) \\
\end{aligned}
\end{equation}


On a vu que, quelle que soit l'étape k considérée dans la production, $ \forall i \in \left\{ {0,1 \ldots 63}\right\}$ les $x^{\lbrack k \rbrack}_{i}$ sont des équations linéaires des $x_i$. De plus dans le calcul des bits de suite chiffrante détaillé ci-dessus, on a uniquement des additions et des multiplications d'ordre 2 des $x^{\lbrack k \rbrack}_{i}$. Ceci montre bien que les bits de $z$ peuvent s'exprimer par des équations quadratiques en les $x_i$.


%% QUESTION 5
\section{Expression des équations quadratiques}
Expression des équations quadratiques en les $x_i$ des bits de la suite chiffrante en sage, cf. a5-2.sage

%% QUESTION 6
\section{Monomes apparaîssant dans les équations quadratiques}

Dans les équations précédentes, les $x_i^{\lbrack k \rbrack }$ représentent les différents états des LFSR intermédiaires 1, 2 et 3. Ces équations sont donc des combinaisons des $x_i$ et celles-ci présenteront un nombre fini de monômes. Les $x_i$ sont inconnus sauf trois d'entre eux que l'on sait égaux à 1.

On va utiliser le coefficient binomial afin de savoir combien de combinaisons de taille k sont possibles dans un ensemble de taille n.

\begin{equation}
  \begin{aligned}
    \binom{n}{k}=\frac{n!}{k!(n-k)!}\\
  \end{aligned}
\end{equation}

Les monômes recherchés étant de degré au moins 2, on pose k = 2, et on peut réduire le calcul ainsi.

\begin{equation}
  \begin{aligned}
    \binom{n}{2}=\frac{n!}{2!(n-2)!}=\frac{(n-1)*n}{2}\\
  \end{aligned}
\end{equation}

Enfin, les monômes seront composés de $x_i$ issus d'un même LFSR (combinaisons des états d'un même LFSR entre eux). Les n correspondront donc aux tailles respectives des LFSR intermédiaires: 19, 22 et 23.

\begin{equation}
  \begin{aligned}
    \binom{19}{2}=\frac{18*19}{2}=171\ \ \ \binom{22}{2}&=\frac{21*22}{2}=231\ \ \ \binom{23}{2}=\frac{22*23}{2}=253\\
    655&=171+231+253\\
  \end{aligned}
\end{equation}

Les bits de suites chiffrante z s'expriment donc par des équations quadratiques, dont les monômes de degré maximum 2 seront au plus au nombre de 655.


%% QUESTION 7
\section{Linéarisation des équations}
Linéarisation des équations en sage, cf. a5-2.sage

%% QUESTION 8
\section{Application}
Application de la linéarisation des équations en sage sur une suite de 700 bits, cf. a5-2.sage

%% QUESTION 9
\section{Question 9}


Soit $A_1$ la matrice de rétroaction du $LFSR_1$. 
Montrons par récurrence que lors de l'étape d'initialisation, après la première boucle for, l'état du $LFSR_1$ est égal à
\begin{equation*}
X = \sum_{i=0}^{63} A_1^i \colvec{4}{0}{\vdots}{0}{K_{63-i}} \\
  = \colvec{4}{0}{\vdots}{0}{K_{63}} + A_1 \colvec{4}{0}{\vdots}{0}{K_{62}} + \cdots + A_1^{63}  \colvec{4}{0}{\vdots}{0}{K_{0}}
\label{eq:sn}
\end{equation*}

Après le premier passage par la première boucle for on a:

\begin{equation*}
X = \colvec{4}{0}{\vdots}{0}{K_{0}} \\
\label{eq:sn}
\end{equation*}

Après le deuxième passage par la première boucle for on a:
\begin{equation*}
X = A_1 \colvec{4}{0}{\vdots}{0}{K_{0}} + \colvec{4}{0}{\vdots}{0}{K_{1}} \\
= \sum_{i=0}^{1} A_1^i \colvec{4}{0}{\vdots}{0}{K_{1-i}}
\label{eq:sn}
\end{equation*}

Supposons pour $k \geq 1$ qu'au $k^{ieme}$ passage par la boucle on ai:
\begin{equation*}
X = \sum_{i=0}^{k} A_1^i \colvec{4}{0}{\vdots}{0}{K_{k-i}}\\
\label{eq:sn}
\end{equation*}

Alors au passage suivant par la boucle on a:

\begin{align*}
X &= A_1 \sum_{i=0}^{k} A_1^i \colvec{4}{0}{\vdots}{0}{K_{k-i}} + \colvec{4}{0}{\vdots}{0}{K_{k+1}} = \sum_{i=1}^{k+1} A_1^i \colvec{4}{0}{\vdots}{0}{K_{k+1-i}} + \colvec{4}{0}{\vdots}{0}{K_{k+1}} \\
X &= \sum_{i=0}^{k+1} A_1^i \colvec{4}{0}{\vdots}{0}{K_{k+1-i}}
\label{eq:sn}
\end{align*}

On obtient donc bien par récurrence que 
\begin{equation*}
X = \sum_{i=0}^{63} A_1^i \colvec{4}{0}{\vdots}{0}{K_{63-i}} \\
\label{eq:sn}
\end{equation*}


De même, le premier passage par la deuxième boucle for on a:
\begin{equation*}
R_1 = A_1 X + \colvec{4}{0}{\vdots}{0}{IV_{0}} = A_1^1 X + \sum_{i=0}^{0} A_1^i \colvec{4}{0}{\vdots}{0}{IV_{0-i}} \\
\label{eq:sn}
\end{equation*}

Pour $k \geq 0 $, on suppose:

\begin{equation*}
R_1 = A_1^{k+1} X + \sum_{i=0}^{k} A_1^i \colvec{4}{0}{\vdots}{0}{IV_{k-i}} \\
\label{eq:sn}
\end{equation*}

A l'étape $k+1$ on a alors:
\begin{align*}
R_1 &= A_1 \times ( A_1^{k+1} X + \sum_{i=0}^{k} A_1^i \colvec{4}{0}{\vdots}{0}{IV_{k-i}}) + \colvec{4}{0}{\vdots}{0}{IV_{k+1}} \\
&= A_1^{k+2} X + \sum_{i=0}^{k} A_1^{i+1} \colvec{4}{0}{\vdots}{0}{IV_{k-i}} + \colvec{4}{0}{\vdots}{0}{IV_{k+1}} \\
&= A_1^{k+2} X + \sum_{i=0}^{k+1} A_1^{i} \colvec{4}{0}{\vdots}{0}{IV_{k+1-i}}
\label{eq:sn}
\end{align*}

On obtient bien par récurrence qu'après la deuxième boucle for, l'état du $LFSR_1$ est

\begin{equation*}
A_1^{22} X + \sum_{i=0}^{21} A_1^i \colvec{4}{0}{\vdots}{0}{IV_{21-i}} \\
\label{eq:sn}
\end{equation*}


%% QUESTION 10
\section{Question 10}

En notant $A_2$, $A_3$ et $A_4$ les matrices de rétroactions des $LFSR_2$, $LFSR_3$ et $LFSR_4$, on trouve de même qu'après la première boucle for, les états respectifs des $LFSR_2$, $LFSR_3$ et $LFSR_4$ sont 
\begin{align*}
X_2 &= \sum_{i=0}^{63} A_2^i \colvec{4}{0}{\vdots}{0}{K_{63-i}} \\
X_3 &= \sum_{i=0}^{63} A_3^i \colvec{4}{0}{\vdots}{0}{K_{63-i}} \\
X_4 &= \sum_{i=0}^{63} A_4^i \colvec{4}{0}{\vdots}{0}{K_{63-i}} \\
\label{eq:sn}
\end{align*}

Après la deuxième boucle for, leurs états respectifs sont:

\begin{align*}
A_2^{22} X_2 + \sum_{i=0}^{21} A_2^i \colvec{4}{0}{\vdots}{0}{IV_{21-i}} \\
A_3^{22} X_3 + \sum_{i=0}^{21} A_3^i \colvec{4}{0}{\vdots}{0}{IV_{21-i}} \\
A_4^{22} X_4 + \sum_{i=0}^{21} A_4^i \colvec{4}{0}{\vdots}{0}{IV_{21-i}}
\end{align*}

%% QUESTION 11
\section{Question 11}
Notons $R_{i,j}$ l'état du registre i après la phase d'initialisation ayant produit les suites $z_j$.\\

Puisque l'IV de $z_0$ est nul, et d'après les questions 9 et 10 on a
\begin{align*}
R_{1,0} &= A_1^{22} X \\
R_{2,0} &= A_2^{22} X_2\\
R_{3,0} &= A_3^{22} X_3\\
R_{4,0} &= A_4^{22} X_4
\end{align*}
Pour $z_1$ seul $IV_{21}$ est non null, donc
\begin{align*}
R_{1,1} = A_1^{22} X + \colvec{4}{0}{\vdots}{0}{1} = R_{1,0} + \colvec{4}{0}{\vdots}{0}{1}\\
R_{2,1} = A_2^{22} X_2 + \colvec{4}{0}{\vdots}{0}{1} = R_{2,0} + \colvec{4}{0}{\vdots}{0}{1} \\
R_{3,1} = A_3^{22} X_3 + \colvec{4}{0}{\vdots}{0}{1} = R_{3,0} + \colvec{4}{0}{\vdots}{0}{1}\\
R_{4,1} = A_4^{22} X_4 + \colvec{4}{0}{\vdots}{0}{1} = R_{4,0} + \colvec{4}{0}{\vdots}{0}{1}
\end{align*}
D'où
\begin{align*}
R_{1,1} &= (x_0, \cdots ,x_{17}, x_{18}+1)\\
R_{2,1} &= (x_{19}, \cdots ,x_{39}, x_{40}+1)\\
R_{3,1} &= (x_{41}, \cdots ,x_{17}, x_{63}+1)\\
R_{4,1} &= R_{4,0} + \colvec{4}{0}{\vdots}{0}{1}
\end{align*}
Et enfin 
\begin{align*}
R_{1,1} &= R_{1,0} + A_1 \colvec{5}{0}{\vdots}{0}{1}{0} 
        = R_{1,0} + \colvec{6}{0}{\vdots}{0}{1}{0}{0} 
        = (x_0, \cdots ,x_{16}+1,x_{17}, x_{18})\\
R_{2,1} &= R_{2,0} + A_2\colvec{5}{0}{\vdots}{0}{1}{0} 
        = R_{2,0} + \colvec{6}{0}{\vdots}{0}{1}{0}{0}
        = (x_{19}, \cdots ,x_{38}+1,x_{37}, x_{40})\\
R_{3,1} &= R_{3,0} + A_3 \colvec{5}{0}{\vdots}{0}{1}{0} 
        = R_{3,0} + \colvec{6}{0}{\vdots}{0}{1}{0}{0}
        = (x_{41}, \cdots ,x_{61}+1,x_{62}, x_{63}) \\
R_{4,1} &= R_{4,0} + A_4 \colvec{5}{0}{\vdots}{0}{1}{0}= 
          R_{4,0} + \colvec{6}{0}{\vdots}{0}{1}{0}{0}
\end{align*}



%% QUESTION 12
\section{Attaque permettant de retrouver la clef K à partir de $z_0, z_1, z_2$}
Afin de trouver les $x_i$ on aura, comme dans la question 8, à créer trois matrices (une pour chacune des suites $z_i$)
chacune possédant 228 équations linéaires en les 655 monômes (on suppose toujours $R_4$ connu, ainsi que les bits fixés à la fin de l'initialisation).
On a donc $684 = 3 \times 228$ équations linéaires et 655 inconnues, on peut résoudre le système et retrouver les $(x_i)_{0\leq i \leq 63}$.
\paragraph{}
Connaissant ces $x_i$ on pourra retrouver la clef, comme dans la question 9, en résolvant un système linéaire de $81$ équations à 64 inconnues.
En effet en utilisant l'IV nul produisant la suite $z_0$, on a 19 équations données par le registre $R_1$, 22 équations données par le registre $R_2$, 23 équations données par le registre $R_3$ et 17 équations données par le registre $R_4$.
Donc au total 81 équations linéairement dépendantes des bits de la clef K.

La résolution de ce système nous permet de retrouver la clef.


%% QUESTION 13
\section{Implementation de l'attaque}

%% QUESTION 14
\section{Retrouver la clef sans connaître $R_4$}

Si l'on ne connaît pas le contenu du registre $R_4$, on ne peut pas calculer
les équations linéaires (ni les équations quadratiques) en les $(x_i)_{0\leq i \leq 63}$. \\

En effet sans connaître l'état du registre $R_4$ on ne peut calculer $m$,
ni savoir quels bits des registres $R_1$, $R_2$ et $R_3$ seront mis à jour.\\

La seule façon que nous avons de retrouver $R_4$ est en testant toutes les valeurs possibles (attaque par force brute),
 jusqu'à trouver un état qui produit bien la bonne suite chiffrante en sortie, après avoir calculé
les $(x_i)_{0\leq i \leq 63}$ et la clef de chiffrement.\\

L'attaque précédente, en supposant connu le registre $R_4$ avait une complexité en $O(L^²)$
où L est le nombre de monomes apparaissant dans les équations. En effet l'étape la plus coûteuse
de l'attaque est le calcul de la matrice représentant les équations linéarisées.\\

En utilisant le paradoxe des anniversaires on peut espérer trouver le bon $R_4$ au bout de
$2^{17/2}$ essais (on a 17 bits à deviner).
On multiplie donc le temps de l'attaque par $2^{9}$. Sachant que notre implémentation retrouve les bits de la clef
secrète en $\approx 9$ secondes, cela devrait nous prendre environ 1h20 pour retrouver la clef sans connaître $R_4$

\begin{align*}
t & = \frac{2^9 \times 9}{3600}\\ 
& = 1.28 \mbox{ heures}\\
& = 1 \mbox{ heure et }20\mbox{ minutes}\\
\end{align*}



%% QUESTION 15
\section{Améliorations dans la conception d'A5/2}
Plusieurs changements au protocole pourraient permettre de le protéger contre
l'attaque vue précédement:
\begin{itemize}
\item{Initialiser les registres avec des valeurs secrètes (ou au moins non
  nulles).}
\item{Complexifier les équations reliant les bits de la suite chiffrante et
  l'état des registres après l'initialisation. Ceci peut se faire en
  choisissant une fonction booléenne de plus haut degré que la fonction Maj.}
\item{Ne pas faire dépendre la mise à jour des registres $R_1$, $R_2$ et
  $R_3$ uniquement du registre $R_4$, il serait en fait plus judicieux de
  calculer la fonction Maj sur des bits de $R_1$, $R_2$ et $R_3$, puisque ceux-ci ont des caractères moins prévisibles que $R_4$ (ils ne sont pas mis à jour à chaque passage par la fonction A5/2-step). }
\end{itemize}

\nocite{*}


\end{document}
