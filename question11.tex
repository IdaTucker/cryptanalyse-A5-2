\section{Expression des registres produisant $z_1$ et $z_2$ en fonction des $x_i$}
Notons $R_{i,j}$ l'état du registre i après la phase d'initialisation ayant produit les suites $z_j$.\\

Puisque l'IV de $z_0$ est nul, et d'après les questions 9 et 10 on a
\begin{align*}
R_{1,0} &= A_1^{22} X \\
R_{2,0} &= A_2^{22} X_2\\
R_{3,0} &= A_3^{22} X_3\\
R_{4,0} &= A_4^{22} X_4
\end{align*}
Pour $z_1$ seul $IV_{21}$ est non null, donc
\begin{align*}
R_{1,1} = A_1^{22} X + \colvec{4}{0}{\vdots}{0}{1} = R_{1,0} + \colvec{4}{0}{\vdots}{0}{1}\\
R_{2,1} = A_2^{22} X_2 + \colvec{4}{0}{\vdots}{0}{1} = R_{2,0} + \colvec{4}{0}{\vdots}{0}{1} \\
R_{3,1} = A_3^{22} X_3 + \colvec{4}{0}{\vdots}{0}{1} = R_{3,0} + \colvec{4}{0}{\vdots}{0}{1}\\
R_{4,1} = A_4^{22} X_4 + \colvec{4}{0}{\vdots}{0}{1} = R_{4,0} + \colvec{4}{0}{\vdots}{0}{1}
\end{align*}
D'où
\begin{align*}
R_{1,1} &= (x_0, \cdots ,x_{17}, x_{18}+1)\\
R_{2,1} &= (x_{19}, \cdots ,x_{39}, x_{40}+1)\\
R_{3,1} &= (x_{41}, \cdots ,x_{17}, x_{63}+1)\\
R_{4,1} &= R_{4,0} + \colvec{4}{0}{\vdots}{0}{1}
\end{align*}
Et enfin 
\begin{align*}
R_{1,2} &= R_{1,0} + A_1 \colvec{5}{0}{\vdots}{0}{0}{1} 
        = R_{1,0} + \colvec{6}{0}{\vdots}{0}{0}{1}{0} 
        = (x_0, \cdots ,x_{16},x_{17}+1, x_{18})\\
R_{2,2} &= R_{2,0} + A_2\colvec{5}{0}{\vdots}{0}{0}{1} 
        = R_{2,0} + \colvec{6}{0}{\vdots}{0}{0}{1}{0}
        = (x_{19}, \cdots ,x_{38},x_{37}+1, x_{40})\\
R_{3,2} &= R_{3,0} + A_3 \colvec{5}{0}{\vdots}{0}{0}{1} 
        = R_{3,0} + \colvec{6}{0}{\vdots}{0}{0}{1}{0}
        = (x_{41}, \cdots ,x_{61},x_{62}+1, x_{63}) \\
R_{4,2} &= R_{4,0} + A_4 \colvec{5}{0}{\vdots}{0}{0}{1}= 
          R_{4,0} + \colvec{6}{0}{\vdots}{0}{0}{1}{0}
\end{align*}

