\section{Monomes apparaîssant dans les équations quadratiques}

Dans les équations précédentes, les $x_i$ représentent les diffèrents états, inconnus sauf 3 d'entre eux que l'on sait égaux à 1, des LFSR intermédiaires 1, 2 et 3. Ces équations sont donc des combinaisons des $x_i$ et celles-ci présenteront un nombre fini de monômes.

On va utiliser le coefficient binomial afin de savoir combien de combinaisons de taille k sont possibles dans un ensemble de taille n.

\begin{equation}
  \begin{aligned}
    \binom{n}{k}=\frac{n!}{k!(n-k)!}\\
  \end{aligned}
\end{equation}

Les monômes recherchés étant de degré au moins 2, on pose k = 2, et on peut réduire le calcul ainsi.

\begin{equation}
  \begin{aligned}
    \binom{n}{2}=\frac{n!}{2!(n-2)!}=\frac{(n-1)*n}{2}\\
  \end{aligned}
\end{equation}

Enfin, les monômes seront composés de $x_i$ issus d'un même LFSR (combinaisons des états d'un même LFSR entre eux). Les n correspondront donc aux tailles respectives des LFSR intermédiaires: 19, 22 et 23.

\begin{equation}
  \begin{aligned}
    \binom{19}{2}=\frac{18*19}{2}=171\ \ \ \binom{22}{2}&=\frac{21*22}{2}=231\ \ \ \binom{23}{2}=\frac{22*23}{2}=253\\
    655&=171+231+253\\
  \end{aligned}
\end{equation}

Les bits de suites chiffrante z s'expriment donc par des équations quadratiques, dont les monômes de degré maximum 2 seront au plus au nombre de 655.
