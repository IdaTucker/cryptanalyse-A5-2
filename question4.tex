\section{Expression des bits de z comme équations quadratiques en les $x_i$: théorique}
\begin{table}[h!]
\centering
\begin{tabular}{l l l|r}
a & b & c & Maj(a,b,c) \\\hline
0 & 0 & 0 & 0  \\
0 & 0 & 1 & 0 \\
0 & 1 & 0 & 0 \\
0 & 1 & 1 & 1 \\
1 & 0 & 0 & 0  \\
1 & 0 & 1 & 1 \\
1 & 1 & 0 & 1 \\
1 & 1 & 1 & 1 \\

\end{tabular}
\caption{\label{tab:widgets}Table de vérité de Maj}
\end{table}

On voit, d'après sa table de véritée, que la forme algébrique normale de la fonction booléenne Maj est:
$$f(a,b,c) = ab \oplus ac \oplus bc$$
Au $(99+i)^{ième}$ passage par la fonction A5/2-step dans la production, on produit le $(i)^{ième}$ bit de la suite chiffrante: 

\begin{equation}
\begin{aligned}
  &z_i =& x^{\lbrack i + 99\rbrack}_0 + \mbox{Maj}(x^{\lbrack i+99\rbrack}_{3}, x^{\lbrack i+99\rbrack}_{4} + 1, x^{\lbrack i+99\rbrack}_{6})) \\
  & &+ x^{\lbrack i + 99\rbrack}_{19} + \mbox{Maj}(x^{\lbrack i+99\rbrack}_{27}, x^{\lbrack i+99\rbrack}_{24} + 1, x^{\lbrack i+99\rbrack}_{31})) \\
  & &+ x^{\lbrack i + 99\rbrack}_{41} + \mbox{Maj}(x^{\lbrack i+99\rbrack}_{45}, x^{\lbrack i+99\rbrack}_{50} + 1, x^{\lbrack i+99\rbrack}_{47})) \\
\end{aligned}
\end{equation}

On peut remplaçer la fonction Maj par sa forme normale algébrique et on obtient:

\begin{equation}
\begin{aligned}
  &z_i =& x^{\lbrack i + 99\rbrack}_0 + (x^{\lbrack i+99\rbrack}_{3}( x^{\lbrack i+99\rbrack}_{4} + 1) + x^{\lbrack i+99\rbrack}_{3} x^{\lbrack i+99\rbrack}_{6} + ( x^{\lbrack i+99\rbrack}_{4} + 1) x^{\lbrack i+99\rbrack}_{6}) \\  
  & &+ x^{\lbrack i + 99\rbrack}_{19} + (x^{\lbrack i+99\rbrack}_{27}(x^{\lbrack i+99\rbrack}_{24} + 1) + x^{\lbrack i+99\rbrack}_{27} x^{\lbrack i+99\rbrack}_{31} + (x^{\lbrack i+99\rbrack}_{24} + 1)x^{\lbrack i+99\rbrack}_{31}) \\  
  & &+ x^{\lbrack i + 99\rbrack}_{41} + (x^{\lbrack i+99\rbrack}_{45} (x^{\lbrack i+99\rbrack}_{50} + 1) + x^{\lbrack i+99\rbrack}_{45} x^{\lbrack i+99\rbrack}_{47} + (x^{\lbrack i+99\rbrack}_{50} + 1)x^{\lbrack i+99\rbrack}_{47}) \\
\end{aligned}
\end{equation}


On a vu que, quelle que soit l'étape k considérée dans la production, $ \forall i \in \left\{ {0,1 \ldots 63}\right\}$ les $x^{\lbrack k \rbrack}_{i}$ sont des équations linéaires des $x_i$. De plus dans le calcul des bits de suite chiffrante détaillé ci-dessus, on a uniquement des additions et des multiplications d'ordre 2 des $x^{\lbrack k \rbrack}_{i}$. Ceci montre bien que les bits de $z$ peuvent s'exprimer par des équations quadratiques en les $x_i$.
