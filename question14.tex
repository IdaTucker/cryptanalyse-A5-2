\section{Retrouver la clef sans connaître $R_4$}

Si l'on ne connaît pas le contenu du registre $R_4$, on a 81 inconnues $(x_i)_{0\leq i \leq 80}$. \\
Pour la première étape de notre attaque, i.e. retrouver les $x_i$ on aura cette fois-ci d'avantage de monomes à considérer.

\begin{align*}
    \binom{19}{2}=171\ \ \ \binom{22}{2}&=231\ \ \ \binom{23}{2}=253\ \ \ \binom{17}{2}=136\\
    791&=171+231+253+136\\
\end{align*}

On a 791 monomes linéarisés de degré 1. On a donc besoin d'au moins 791 équations linéaires en les $x_i$, soit 791 bits de suite chiffrante.

Avec 4 suites chiffrantes $z_0$, $z_1$, $z_2$, $z_3$ avec des IV respectifs de $(0,\cdots ,0,0)$, $(0,\cdots ,0,1)$, $(0,\cdots ,1,0)$ et $(0,\cdots ,1,0,0)$ on obtient 912 équations linéarisées en les 791 monomes de degré au plus 2.

On peut ainsi récupérer les valeurs de $R_1$, $R_2$, $R_3$, $R_4$ après la phase d'initialisation c'est à dire les $(x_i)_{0\leq i \leq 80}$. 

Ensuite grâce aux 81 équations reliant les $x_i$ aux bits de la clef (obtenus comme dans les questions 9 et 10 à partir d'un IV null), on obtient encore une fois 81 équations à 64 équations.

Cette fois-ci on aura 4 bits de la clef à deviner, les 4 qui ont été fixés à la fin de l'initialisation. On essaie donc de résoudre le système avec différentes valeurs de $R_{1,3}$, $R_{2,5}$, $R_{3,4}$, $R_{4,6}$ jusqu'à obtenir un système qui admet une solution.
