\section{Question 2}


Avant le $1^{er}$ passage par la fonction A5/2-step on a: 
\begin{itemize}
\item $R_1 = (x_0, \ldots, x_{18})$
\item $R_2 = (x_{19}, \ldots, x_{40})$
\item $R_1 = (x_{41}, \ldots, x_{63})$
\end{itemize}

Notons, au $k^{ième}$ passage par la fonction A5/2-step:
\begin{itemize}
\item $R_1 = (x^{\lbrack k\rbrack}_0, \ldots, x^{\lbrack k \rbrack}_{18})$
\item $R_2 = (x^{\lbrack k \rbrack}_{19}, \ldots, x^{\lbrack k \rbrack}_{40})$
\item $R_3 = (x^{\lbrack k \rbrack}_{41}, \ldots, x^{\lbrack k \rbrack}_{63})$
\end{itemize}

Montrons qu'on peut exprimer le contenu des registres $R_1$, $R_2$, $R_3$ au moyen d'équations linéaires en les $x_i$ par récurrence.

Il est clair qu'à l'étape zero (avant le premier passage par A5/2-step), la propriété est vraie.
Supposons la vraie au rang $k$ pour $k \geq 0$.
On a donc $x^{\lbrack k\rbrack}_0, \ldots, x^{\lbrack k \rbrack}_{63}$ qui s'expriment comme équation linéaire des $x_i$.
On passe alors dans la fonction A5/2-step:
\begin{itemize}
\item Si m = $R_{4,6}$ on met à jour le $LFSR_1$, et $R_1$ devient:
$$R_1 = (x^{\lbrack k+1\rbrack}_0, \ldots, x^{\lbrack k+1 \rbrack}_{18}) = (x^{\lbrack k\rbrack}_1, x^{\lbrack k\rbrack}_2, \ldots, x^{\lbrack k \rbrack}_{18}, x^{\lbrack k\rbrack}_0 + x^{\lbrack k\rbrack}_1 + x^{\lbrack k\rbrack}_2 + x^{\lbrack k\rbrack}_5)$$
\item Si m = $R_{4,13}$ on met à jour le $LFSR_2$, et $R_2$ devient:
$$R_2 = (x^{\lbrack k+1\rbrack}_{19}, \ldots, x^{\lbrack k+1 \rbrack}_{40}) = (x^{\lbrack k\rbrack}_{20},x^{\lbrack k\rbrack}_{21}, \ldots, x^{\lbrack k+1 \rbrack}_{40}, x^{\lbrack k\rbrack}_{19} + x^{\lbrack k\rbrack}_{20})$$
\item Si m = $R_{4,9}$ on met à jour le $LFSR_3$, et $R_3$ devient:
$$R_3 = (x^{\lbrack k+1\rbrack}_{41}, \ldots, x^{\lbrack k+1 \rbrack}_{63}) = (x^{\lbrack k\rbrack}_{42}, x^{\lbrack k\rbrack}_{43}, \ldots, x^{\lbrack k\rbrack}_{63},x^{\lbrack k\rbrack}_{41} + x^{\lbrack k\rbrack}_{42}+ x^{\lbrack k\rbrack}_{43}+ x^{\lbrack k\rbrack}_{56})$$
\end{itemize}

On a donc le dernier élément de chaque registre au $(k+1)^{ième}$ passage qui est une somme des contenus des registre au $k^{ième}$ passage.
C'est bien une équation linéaire en les $x_i$ par linéarité de la somme.
Les autres éléments des registres sont les éléments de l'étape précédente décalés. C'est à dire $x^{\lbrack k+1\rbrack}_{i} = x^{\lbrack k\rbrack}_{i+1}$ pour $ i \neq 18, 40, 63$.

On a bien montré que durant toutes les étapes de la production de suite chiffrante de A5/2, on peut exprimer les contenus des registres $R_1$, $R_2$, $R_3$ au moyen d'équations linéaires en les $x_i$. 